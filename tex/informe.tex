\documentclass[a4paper,12pt]{article}
\usepackage[utf8]{inputenc}
\usepackage{fancyhdr, float, graphicx, caption}
\usepackage{amsmath, amssymb}
\usepackage{bm}
\usepackage[margin=1in]{geometry}
\usepackage{multicol}
\usepackage{titlesec} 


\titleformat{\subsection}[runin]
  {\normalfont\large\bfseries}{\thesubsection}{1em}{}	
\titleformat{\subsubsection}[runin]
  {\normalfont\normalsize\bfseries}{\thesubsubsection}{1em}{}


\pagestyle{fancy}
\renewcommand{\figurename}{Figura}
\renewcommand\abstractname{\textit{Abstract}}

\fancyhf{}
\fancyhead[LE,RO]{\textit{DFS Server}}
\fancyfoot[RE,CO]{\thepage}

%%%%%%%%%%%%%%%%%%%%%%%%%%%%%%%%%%%%%%%%%%%%%%%%%%%%%%%%%%%%

\title{
	%Logo UNR
	\begin{figure}[!h]
		\centering
		\includegraphics[scale=1]{unr.png}
		\label{}
	\end{figure}
	% Pie Logo
	\normalsize
		\textsc{Universidad Nacional de Rosario}\\	
		\textsc{Facultad de Ciencias Exactas, Ingeniería y Agrimensura}\\
		\textit{Licenciatura en Ciencias de la Computación}\\
		\textit{Sistemas Operativos I}\\
	% Título
	\vspace{30pt}
	\hrule{}
	\vspace{15pt}
	\huge
		\textbf{Sistema de archivos distribuído}\\
	\vspace{15pt}
	\hrule{}
	\vspace{30pt}
	% Alumnos/docentes
	\begin{multicols}{2}
	\raggedright
		\large
			\textbf{Alumnos:}\\
		\normalsize
			BORRERO, Paula (P-????)\\
			IVALDI, Ángela (I-????)\\
			MISTA, Agustín (M-6105/1) \\
	\raggedleft
		\large
			\textbf{Docentes:}\\
		\normalsize
			MACHI, Guido\\
			GRINBLAT, Guillermo\\
			DIAZ, José Luis\\
	\end{multicols}
}
%%%%%%%%%%%%%%%%%%%%%%%%%%%%%%%%%%%%%%%%%%%%%%%%%%%%%%%%%%%
\begin{document}
\date{4 de Julio de 2016}
\maketitle

\pagebreak
%----------------------------------------------------------
\section*{Introducción}
	
	Un servidor de archivos distribuido es un componente de software que le ofrece al usuario final las operaciones necesarias para trabajar con un sistema de archivos virtual, aparentemente centralizado, donde todos los archivos parecen estar en una misma ubicación, cuando en realidad es probable que los mismos estén dispersos en varias unidades de disco, o más aun, en varias computadoras.
	
	En éste informe analizaremos las implementaciones tanto en \textbf{C}  como en \textbf{Erlang} de un servidor de archivos distribuido simple, esto incluye profundizar sobre algunas cuestiones de diseño tales como comunicación entre hilos, concurrencia y performance.
	
	
\end{document}